\documentclass[a4paper, french, openany, 12pt]{book}
\usepackage[utf8]{inputenc}
\usepackage[T1]{fontenc}
\usepackage{babel}
\usepackage{pifont}
\usepackage{graphicx}
\usepackage{fullpage}
\usepackage[dvipsnames]{xcolor} % colors
\usepackage{tabto}
\usepackage{stackengine}
\usepackage[skip=20pt]{parskip} % space between paragraphs
\usepackage[hidelinks]{hyperref} % links
\usepackage[framemethod=TikZ]{mdframed} % frames with colored background
\usepackage{titlesec} % exclude numbered sections from table of contents
\usepackage{amssymb} % symbols
\usepackage{multicol} % multiple columns

\def\changemargin#1#2{\list{}{\rightmargin#2\leftmargin#1}\item[]}
\let\endchangemargin=\endlist 
\setcounter{secnumdepth}{-1} % exclude numbered sections from chapter titles

\newcommand{\fullwidthimage}[1]{
  \begin{center}
    \makebox[\textwidth]{\includegraphics[width=\paperwidth]{#1}}
  \end{center}
}

\newcommand{\todo}[1]{
  \begin{color}{RubineRed}
    \texttt{TODO {#1}}
  \end{color}
}

\newcommand\dex{\textcolor{BrickRed}{\textbf{\bsc{Dexterity} - Technical skill}}}
\newcommand\str{\textcolor{DarkOrchid}{\textbf{\bsc{Strength} - Get stuff done}}}
\newcommand\wis{\textcolor{MidnightBlue}{\textbf{\bsc{Wisdom} - Impact}}}
\newcommand\cha{\textcolor{RawSienna}{\textbf{\bsc{Charisma} - Communication \& leadership}}}

\newcommand\xp[1]{\textcolor{Gray}{Expérience: {#1} ans}}

\title{
  \vspace*{-8cm}

  \fullwidthimage{images/cover.jpg}

  \vspace*{5cm}

  \bsc{Référenciel de compétences pour développeurs et développeuses}

  Le framework pour vous positionner et évoluer dans votre carrière.
}

\author{Sébastien Carceles}

\date{}

\begin{document}

\begin{titlepage}
  \maketitle
\end{titlepage}

\frontmatter

\chapter{Avant-propos}

\section*{Objectif de ce référentiel}

Les métiers du développement sont en mutation perpétuelle.
Les technologies évoluent, les méthodes de travail changent, les attentes des clients aussi.
Il est difficile de s'y retrouver et de savoir où l'on se situe dans ce paysage mouvant.

Les entreprises ont besoin de référentiels pour évaluer les compétences de leurs employés.
Les développeur·euses ont besoin de référentiels pour se positionner et évoluer dans leur carrière.
Ce référentiel a pour but de répondre à ces deux besoins.

\section*{Positions}

Les entreprises nomment les postes de leurs employé·es de différentes manières.
Cette nommenclature n'est pas normée : un poste peut être appelé "développeur / développeuse", "ingénieur·e de 
développement", "consultant / consultante", "expert / experte", "architecte", "lead", "manager", etc.
Il peut correspondre à un poste junior, mid-level, senior, lead, manager, etc, et couvrir différents domaines de 
responsabilités, qui diffèrent d'une entreprise à l'autre.

Ce référentiel propose une nomenclature des postes, qu'on appellera "positions",
ainsi qu'une description des compétences attendues pour chaque position.

Dans l'industrie du développement logiciel, les noms de postes ou de positions sont, la plupart du temps, en anglais.
C'est pourquoi ce référentiel utilise des noms en anglais.

\section*{Caractéristiques et compétences}

À chaque position correspond un ensemble de compétences, organisées par caractéristique principale.
Les caractaristiques sont au nombre de 4 :

\subsubsection*{\dex} 

C'est une mesure de la profondeur de la connaissance d'une technologie.

\subsubsection*{\str} 

C'est une mesure de la capacité à résoudre des problèmes.

\subsubsection*{\wis} 

C'est une mesure de la capacité à prendre des décisions et à avoir de l'impact.

\subsubsection*{\cha} 

C'est une mesure de la capacité à influencer et inspirer les autres.

\section*{Positionnement}

Pour être légitime dans une position donnée, il faut en maîtriser toutes les compétences.
Il faut également posséder les compétences des positions précédentes dans le chemin parcouru.

\section*{Évolution de position}

Pour évoluer d'une position à l'autre, il faut naturellement acquérir, consolider et maîtriser toutes les compétences
qui la composent.

\section*{Source}

Ce référentiel est issue du post de blog 
"Sharing Our Engineering Ladder"\footnote{\url{https://dresscode.renttherunway.com/blog/ladder}}
de la société Rent the Runway.

Il a ensuite été traduit, retravaillé et adapté pour le marché français.

\section*{Images}

Photo de couverture par Annie Spratt\footnote{\url{https://unsplash.com/fr/@anniespratt}} sur Unsplash.

\section*{À propos du genre}

Autant que possible, le présent document utilise des termes non genrés.
Dans les cas où c'est nécessaire, le point médian et les pronoms non genrés permettent une meilleure inclusivité.

\section*{Propriété intellectuelle}

\todo{open licence?}

\mainmatter

\part{Les différents parcours}

\fullwidthimage{images/tracks.png}

La carrière du développeur ou de la développeuse peut emprunter plusieurs chemins.
Ils ne sont pas rigoureusement étanches et des passages d'un chemin à l'autre sont possibles.

\subsubsection*{Contributor Track}

Dans tous les cas, une carrière commence avec la première position du premier parcours, nommé "Contributor Track".
Dans ce parcours, au travers des positions qui le composent, la personne acquiert des compétences clés pour être une
excellente contributrice au projet.

\subsubsection*{Specialist Track}

Après avoir atteint la position "Lead Developer" du "Contributor Track", la personne peut choisir de se spécialiser dans
la technique, en choisissant le "Specialist Track".
Si elle reste principalement contributrice, c'est un parcours qui l'amène sur le chemin de l'expertise technique.

\todo{la suite}

\part{Contributor Track}

\chapter{Junior developer}

\xp{0 à 2}

\subsubsection*{\dex} 

Possède une connaissance approfondie des concepts fondamentaux de l'informatique.

Se concentre sur la croissance en tant qu'ingénieur·e, en apprenant les outils, les ressources et les processus 
existants.

\subsubsection*{\str}

Développe ses compétences en productivité en apprenant le contrôle de source, les éditeurs, le système de construction 
et autres outils, ainsi que les meilleures pratiques de test.

Est capable de prendre des sous-tâches bien définies et de les accomplir.

\subsubsection*{\wis}

Développe ses connaissances sur un seul composant de l'architecture du projet.

\subsubsection*{\cha}

Est efficace dans la communication de l'état d'avancement à l'équipe.

Manifeste les valeurs fondamentales de l'entreprise, se concentre sur la compréhension et la mise en pratique de ces 
valeurs.

Accepte gracieusement les retours d'information et apprend de tout ce qu'iel fait.

\chapter{Developer}

\xp{2 à 5}

\subsubsection*{\dex}

Écrit du code correct et propre avec des conseils ; suit systématiquement les meilleures pratiques définies sur le
projet.

Participe à la conception technique des fonctionnalités avec des conseils.

\subsubsection*{\dex}

Fait rarement la même erreur deux fois, commence à se concentrer sur l'acquisition d'une expertise dans un ou plusieurs 
domaines (par exemple : développement dans un langage donné, meilleures pratiques de performance, utilisation efficace 
des bases de données, messagerie, etc.)

Apprend rapidement et progresse régulièrement sans avoir besoin de commentaires significatifs constants de la part 
d'ingénieur·es plus expérimenté·es.

\subsubsection*{\str}

Fait des progrès réguliers sur les tâches ; sait quand demander de l'aide pour se débloquer.

\subsubsection*{\str}

Est capable de prendre en charge des fonctionnalités de petite à moyenne envergure, de la conception technique à 
l'achèvement.

Est capable de hiérarchiser les tâches ; évite de s'enliser dans des détails sans importance et des "discussions sans 
fin".

\subsubsection*{\wis}

Autonome dans au moins un grand domaine du code avec une compréhension globale des autres composants.

\subsubsection*{\wis}

Est capable d'assurer une assistance en cas d'urgence pour leur domaine, y compris les systèmes avec lesquels iel 
n'est pas familier·ère.

\subsubsection*{\cha}

Donne des commentaires opportuns et utiles aux pairs et aux responsables.

Communique les hypothèses et obtient des éclaircissements sur les tâches dès le départ afin de minimiser le besoin de 
retravailler.

\subsubsection*{\cha}

Sollicite les commentaires des autres et est désireux de trouver des moyens de s'améliorer.

Comprend comment son travail s'inscrit dans le projet global et identifie les problèmes liés aux exigences.

\chapter{Senior developer}

\xp{5 à 8}

\subsubsection*{\dex}

Comprend et prend des décisions de conception bien raisonnées et des compromis dans leur domaine.

Est capable de travailler dans d'autres domaines du code avec des conseils.

Ne s'agite pas lors du débogage.

\subsubsection*{\dex}

Démontre une connaissance des tendances de l'industrie, de l'infrastructure du projet et de son système de construction.

\subsubsection*{\str}

Persistant face aux obstacles ; les résout efficacement, en faisant appel à d'autres personnes si nécessaire. 

Nécessite un minimum de direction / supervision.

\subsubsection*{\str}

Prend l'initiative de résoudre les problèmes avant qu'ils ne lui soient assignés. 
Recherche des preuves empiriques par le biais de preuves de concept, de tests et de recherches externes.

Livre des produits complexes à l'équipe de QA / Produit, qu'iel estime bien préparés et exempts de bugs.

\subsubsection*{\wis}

Responsable de bout en bout sur des projets de complexité croissante ; contribue au code commun.

Examine les cas de test et conseille l'équipe de QA / Produit sur l'impact du code adjacent / régression.

Comprend l'activité commerciale afférente à son domaine d'activité.

\subsubsection*{\wis}

Collabore avec le produit et l'analyse et définit des exigences qui tiennent compte des besoins de toutes les parties.

Possède de l'empathie envers l'utilisateur du logiciel qu'iel produit et utilise cette empathie pour orienter la prise 
de décision.

Identifie les problèmes / risques de son propre travail et de celui des autres.

\subsubsection*{\cha}

Communique les décisions techniques par le biais de documents de conception ou de présentations techniques.

Mentore les ingénieur·es juniors par le biais de la collaboration, de l'examen de conception et de l'examen de code. 
Contribue fréquemment aux réunions informelles et aux démonstrations.

\subsubsection*{\cha}

Communique efficacement entre les fonctions.
Est capable de bien travailler avec le produit, le design, l'analyse, etc., si nécessaire.

Identifie de manière proactive les problèmes liés aux exigences (manque de clarté, incohérences, limitations techniques)
pour son propre travail et le travail adjacent, et communique ces problèmes tôt pour aider à corriger le tir.

\chapter{Lead developer}

\xp{6 à 10}

\subsubsection*{\dex}
Expert de référence dans un domaine spécifique du code.

Comprend l'architecture globale de l'ensemble du système.

\subsubsection*{\dex}
Fournit des conseils techniques et donne son avis sur les décisions techniques qui ont un impact sur d'autres équipes ou
sur l'entreprise dans son ensemble. 

Effectue des recherches et propose de nouvelles technologies.

\subsubsection*{\str}

Définit et organise le travail en étapes bien définies pour éviter un livrable monolithique.

Livre régulière dans les délais et travail constant pour faire des estimations précises et les respecter.

\subsubsection*{\str}

Est réputé pour des lancements sans problème.

Est responsable du plan de tests techniques et de performance pour ses projets.

\subsubsection*{\wis}

Prend l'initiative d'identifier et de résoudre des problèmes importants, en se coordonnant avec d'autres personnes sur
des problèmes techniques transversaux.

\subsubsection*{\wis}

Définit l'orientation au niveau du projet / service et influence de manière constante la prise de décision au niveau
technique.

Identifie et aborde de manière proactive la dette technique avant qu'elle ne devienne trop coûtuse à rembourser.

\subsubsection*{\cha}

Aide les autres personnes à s'améliorer grâce à des revues de code, une documentation approfondie, des conseils 
techniques et un mentorat ou en tant que chef technique sur un projet.

\subsubsection*{\cha}
Siège aux comités de prise de décisions techniques ou architecturales, donne des commentaires sur des projets en dehors 
de son domaine principal.

Comprend les compromis entre les besoins techniques, analytiques et produits et propose des solutions qui tiennent 
compte de tous ces besoins.

Identifie et propose des stratégies pour résoudre les problèmes techniques affectant son équipe, communique des normes 
et obtient l'adhésion aux solutions.

\part{Specialist Track}

\chapter{Principal developer}

\xp{10 à 15}

\chapter{Senior principal developer}

\xp{15 à 20}

\chapter{Architect}

\part{Manager Track}

\chapter{Engineering manager}

\chapter{Engineering director}

\chapter{VP of engineering}

\chapter{Exec}

\backmatter

\tableofcontents

\end{document}
