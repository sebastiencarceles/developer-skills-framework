\documentclass[a4paper, french, openany, 12pt]{book}
\usepackage[utf8]{inputenc}
\usepackage[T1]{fontenc}
\usepackage{babel}
\usepackage{pifont}
\usepackage{graphicx}
\usepackage{fullpage}
\usepackage[dvipsnames]{xcolor} % colors
\usepackage{tabto}
\usepackage{stackengine}
\usepackage[skip=20pt]{parskip} % space between paragraphs
\usepackage[hidelinks]{hyperref} % links
\usepackage[framemethod=TikZ]{mdframed} % frames with colored background
\usepackage{titlesec} % exclude numbered sections from table of contents
\usepackage{amssymb} % symbols
\usepackage{multicol} % multiple columns

\def\changemargin#1#2{\list{}{\rightmargin#2\leftmargin#1}\item[]}
\let\endchangemargin=\endlist 
\setcounter{secnumdepth}{-1} % exclude numbered sections from chapter titles

\newcommand{\fullwidthimage}[1]{
  \begin{center}
    \makebox[\textwidth]{\includegraphics[width=\paperwidth]{#1}}
  \end{center}
}

\newcommand\marginfigure[2][]{
  \leavevmode
  \tabto*{-20pt}
  \smash{\belowbaseline[-\ht\strutbox]{\makebox[0pt][r]
  {\includegraphics[#1]{#2}}}}
  \tabto*{\TabPrevPos}
}

\newcommand{\todo}[1]{
  \begin{color}{RubineRed}
    \texttt{TODO {#1}}
  \end{color}
}

\newenvironment{separator}{
  \begin{center}
    \ding{167} 
    \ding{167} 
    \ding{167}
    \ding{167}
    \ding{167}
  \end{center}
}

\newcommand\dex{\textcolor{BrickRed}{\textbf{\bsc{Dex}}}}
\newcommand\str{\textcolor{DarkOrchid}{\textbf{\bsc{Str}}}}
\newcommand\wis{\textcolor{MidnightBlue}{\textbf{\bsc{Wis}}}}
\newcommand\cha{\textcolor{RawSienna}{\textbf{\bsc{Cha}}}}

\newcommand\xp[1]{\textcolor{Gray}{Expérience: {#1} ans}}

\title{
  \vspace*{-8cm}

  \fullwidthimage{images/cover.jpg}

  \vspace*{5cm}

  \bsc{Référenciel de compétences pour développeurs et développeuses}

  Le framework pour vous positionner et évoluer dans votre carrière.
}

\author{Sébastien Carceles}

\date{}

\begin{document}

\begin{titlepage}
  \maketitle
\end{titlepage}

\frontmatter

\chapter{Avant-propos}

\section*{Objectif de ce référentiel}

Le métier de développeur est en mutation perpétuelle.
Les technologies évoluent, les méthodes de travail changent, les attentes des clients aussi.
Il est difficile de s'y retrouver et de savoir où l'on se situe dans ce paysage mouvant.

Les entreprises ont besoin de référentiels pour évaluer les compétences de leurs employés.
Les développeurs ont besoin de référentiels pour se positionner et évoluer dans leur carrière.
Ce référentiel a pour but de répondre à ces deux besoins.

\section*{Nomenclature des postes}

Par ailleurs, les entreprises nomment les postes de leurs employés de différentes manières.
Cette nommenclature n'est pas normée : un poste de développeur peut être appelé "développeur", "ingénieur", 
"consultant", "expert", "architecte", "lead", "manager", etc.
Il peut correspondre à un poste junior, mid-level, senior, lead, manager, etc, et couvrir différents domaines de 
responsabilités, qui diffèrent d'une entreprise à l'autre.

Ce référence propose une nomenclature des postes de développeurs, qu'on appellera "positions",
ainsi qu'une description des compétences attendues.

Dans l'industrie du développement logiciel, les noms de postes ou de positions sont, la plupart du temps, en anglais.
C'est pourquoi ce référentiel utilise des noms en anglais.

\section*{Caractéristiques}

À chaque position correspond un ensemble de compétences, exprimées selon des caractéristiques.

\subsubsection*{Dexterity: technical skill} 

La \dex\ est une mesure de la profondeur de la connaissance d'une technologie.

\subsubsection*{Strength: get stuff done} 

La \str\ est une mesure de la capacité à résoudre des problèmes.

\subsubsection*{Wisdom: impact} 

La \wis\ est une mesure de la capacité à prendre des décisions et à avoir de l'impact.

\subsubsection*{Charisma: leadership} 

Le \cha\ est une mesure de la capacité à influencer et inspirer les autres.

\section*{Source}

Ce référentiel est issue du post de blog 
"Sharing Our Engineering Ladder"\footnote{\url{https://dresscode.renttherunway.com/blog/ladder}}
de la société Rent the Runway.

Il a ensuite été traduit et retravaillé et adapté pour le marché français.

\section*{Propriété intellectuelle}

\todo{open licence?}

\todo{images rights}

\mainmatter

\part{Les différents chemins}

\part{Contributor Track}

\chapter{Junior developer}

\xp{0 à 2}

\subsubsection*{\dex} 

Possède une connaissance approfondie des concepts fondamentaux de l'informatique.

Se concentre sur la croissance en tant qu'ingénieur, en apprenant les outils, les ressources et les processus existants.

\subsubsection*{\str}

Développe ses compétences en productivité en apprenant le contrôle de source, les éditeurs, le système de construction 
et autres outils, ainsi que les meilleures pratiques de test.

Est capable de prendre des sous-tâches bien définies et de les accomplir.

\subsubsection*{\wis}

Développe ses de connaissances sur un seul composant de l'architecture du projet.

\subsubsection*{\cha}

Est efficace dans la communication de l'état d'avancement à l'équipe.

Manifeste les valeurs fondamentales de l'entreprise, se concentre sur la compréhension et la mise en pratique de ces 
valeurs.

Accepte gracieusement les retours d'information et apprend de tout ce qu'il fait.

\chapter{Mid-level developer}

\xp{2 à 5}

\chapter{Senior developer}

\xp{5 à 8}

\chapter{Lead developer}

\xp{6 à 10}

\part{Specialist Track}

\chapter{Principal developer}

\xp{10 à 15}

\chapter{Senior principal developer}

\xp{15 à 20}

\chapter{Architect}

\part{Manager Track}

\chapter{Engineering manager}

\chapter{Engineering director}

\chapter{VP of engineering}

\chapter{Exec}

\backmatter

\tableofcontents

\end{document}
